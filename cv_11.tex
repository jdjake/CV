%%%%%%%%%%%%%%%%%%%%%%%%%%%%%%%%%%%%%%%%%
% Friggeri Resume/CV
% XeLaTeX Template
% Version 1.2 (3/5/15)
%
% This template has been downloaded from:
% http://www.LaTeXTemplates.com
%
% Original author:
% Adrien Friggeri (adrien@friggeri.net)
% https://github.com/afriggeri/CV
%
% License:
% CC BY-NC-SA 3.0 (http://creativecommons.org/licenses/by-nc-sa/3.0/)
%
% Important notes:
% This template needs to be compiled with XeLaTeX and the bibliography, if used,
% needs to be compiled with biber rather than bibtex.
%
%%%%%%%%%%%%%%%%%%%%%%%%%%%%%%%%%%%%%%%%%

\documentclass{friggeri-cv} % Add 'print' as an option into the square bracket to remove colors from this template for printing

% \addbibresource{bibliography.bib} % Specify the bibliography file to include 

\begin{document}

\header{jacob}{denson}{Mathematics Student} % Your name and current job title/field

%----------------------------------------------------------------------------------------
%	SIDEBAR SECTION
%----------------------------------------------------------------------------------------

\begin{aside} % In the aside, each new line forces a line break
\section{Research Interests}
Harmonic Analysis, Geometric Measure Theory, Additive Combinatorics.
\section{Contact Information}
%5692 King Street,
%Vancouver, BC, Canada
%V6T1K8
%~
%+1 (780) 862-7414
%~
Additional Contact Info Redacted For Web Version
\href{mailto:denson@math.ubc.ca}{denson@math.ubc.ca}
\section{Websites}
\href{https://github.com/jdjake}{{\bf Github Profile}: jdjake}
~
\href{http://stackoverflow.com/users/2601483/jacob-denson}{{\bf Stack Overflow Profile:} jacob-denson}
~
\href{https://jdjake.github.io/}{https://jdjake.github.io/}
\section{Languages}
English, Elementary German, Very Rusty Elementary Chinese
Python, Perl, C++, C, C\#, Matlab, HTML, Javascript, Latex (This resume is proof!)
\end{aside}

%----------------------------------------------------------------------------------------
%   KEY SKILLS
%----------------------------------------------------------------------------------------

\section{Summary}

I am a masters student at the University of British Columbia, applying my strong and diverse foundation in mathematical knowledge to do research in the harmonic analysis research group, studying continuous variants of discrete configuration avoidance problems emerging from additive combinatorics. My previous work in theoretical computing science has given me a strong knowledge of the algorithmic viewpoint of problems, which gives me a fresh perspective on classical ideas in the field. I am currently working on the problem of finding high dimensional subsets of intervals avoiding solutions to polynomial equations.

%----------------------------------------------------------------------------------------
%	WORK EXPERIENCE SECTION
%----------------------------------------------------------------------------------------

\pagebreak[3]
\section{Talks\footnote{Notes for my talks can be found on my website: \href{https://jdjake.github.io/}{https://jdjake.github.io/}}} % \& Publications}

\begin{entrylist}
  
%------------------------------------------------

\entry
{2017}
{Graduate Seminar}
{University of British Columbia}
{\emph{Proofs in Three Bits or Less}\\
An hour talk introducing nonspecialists to the theory of probabilistically checkable proofs, and PCP theory. By changing the language by which we discuss the theory from accessing random bits from a string, to `playing a game of 20 questions', I introduced a novel way to discuss the theory which avoids the technicalities of the field, making the talk accessible to students without any background in theoretical computing science.}

%------------------------------------------------

%------------------------------------------------

\entry
{2016}
{Noncommutative Harmonic Analysis Class}
{University of Alberta, Canada}
{\emph{Why Physicists Care About The Fourier-Stieltjes Transform}\\
A 20 minute talk emphasizing the naturality of the generalization of the Fourier transform to the Fourier-Stieltjes transform by proving the weak $*$ density of $L^1(G)$ in $M(G)$, and discussing why this matters.}

\entry
{2016}
{Noncommutative Harmonic Analysis Class}
{University of Alberta, Canada}
{\emph{A Brief Respite In Abelian Analysis}\\
A 20 minute talk introducing the abstract Fourier transform on abelian locally compact groups, and discussing the generalization of the Poincare summation formula to this domain, which hints at the depth of Pontrayagin duality.}

%------------------------------------------------

\entry
{2016}
{CUMC Conference}
{University of Victoria, Vancouver Island}
{\emph{On Molecular Gases and the Natural Numbers}\\
A talk introducing Ergodic theory to undergraduate students, and emphasizing its relation to a variety of problems in mathematics, especially number theory.}

%------------------------------------------------

\entry
{2016}
{Algebraic Topology Graduate Class}
{University of Alberta, Canada}
{\emph{Vector Fields, Hex, and Jordan Curves}\\
A 20 minute talk on the Brouwer fixed-point theorem, emphasizing the intuitive vector field interpretation of the theorem, and discussing how the fixed-point theorem relates to the combinatorial game of hex, reflecting the nice interweaving of discrete and point-set methods in algebraic topology.}

\end{entrylist}

%\newgeometry{margin=1in,top=1cm,right=1cm,bottom=1cm}

\begin{entrylist}

%------------------------------------------------
    
\entry
{2015}
{Microsoft Intern Talks}
{Microsoft Campus, Redmond}
{\emph{Category Theory for Computer Programmers}\\
My original talk on category theory, shortened to a 20 minutes talk, and edited to reduce mathematical prerequisites and to emphasize the practical uses for the average programmer, as a talk in the weekly talk seminar for inerns I ran about various interesting topics in computing science.}

%------------------------------------------------

\entry
{2015}
{Honours Computing Science Seminar}
{University of Alberta}
{\emph{Category Theory and its relation to Computing Science}\\
an hour-long talk introducing the subject to Honours computing scientists and emphasizing its relation to the Curry Howard isomorphism.}

%------------------------------------------------

\entry
{2014}
{NLP Research Group}
{University of Alberta}
{\emph{Cognates for Reconstruction of Native American Language groups}\\
a 20 minute talk emphasizing my work over the summer and explaining the organization method and SVM classification method for identifying cognates.}

%------------------------------------------------

\entry
{2013}
{RLAI Tea Time Talks}
{University of Alberta}
{\emph{Room Abstraction in Sokoban}\\
a 15 minute talk introducing the game of Sokoban, its combinatorial issues, and room abstraction as an aid to attacking the game.}

\end{entrylist}

\section{Experience}

\subsection{Selected Mathematics Courses (3.96 Math GPA, 3.8 General GPA)\footnote{An asterix indicates a course I plan to take in the winter semester}}

\begin{entrylist}

%------------------------------------------------

\entries
{FUNCTIONAL ANALYSIS}
{
\begin{itemize}
    \setlength\itemsep{-1em}
    \item Banach Spaces (A)\\
    \item Operator Algebras (A+)\\
    \item Euclidean Harmonic Analysis*\\
    \item Abstract Harmonic Analysis (A+)\\
    \item Partial Differential Equations
\end{itemize}
}

%------------------------------------------------

\entries
{COMPLEX ANALYSIS}
{
\begin{itemize}
    \setlength\itemsep{-1em}
    \item Complex Variables (A-)\\
    \item Modular Forms (A)
\end{itemize}
}

%------------------------------------------------

\entries
{ALGEBRA}
{
\begin{itemize}
    \setlength\itemsep{-1em}
    \item Galois Theory (A)\\
    \item Representation Theory of Lie Algebras (B+)
\end{itemize}
}

%------------------------------------------------

\entries
{TOPOLOGY}
{
\begin{itemize}
    \setlength\itemsep{-1em}
    \item Topology (A+)\\
    \item Algebraic Topology (A+)
\end{itemize}
}

\entries
{DISCRETE MATHEMATICS}
{
\begin{itemize}
    \setlength\itemsep{-1em}
    \item Combinatorial Optimization (A)\\
    \item Fourier Analysis of Boolean Functions (A+)\\
    \item Analytic Number Theory*
\end{itemize}
}

%------------------------------------------------

\end{entrylist}

%------------------------------------------------

\begin{entrylist}

\entries
{PROBABILITY THEORY}
{
\begin{itemize}
    \setlength\itemsep{-1em}
    \item Stochastic Processes (A+)\\
    \item Multi Armed Bandits (A+)\\
    \item Brownian Motion and Stochastic Integration
\end{itemize}
}

%------------------------------------------------

\entries
{GEOMETRY}
{
\begin{itemize}
    \setlength\itemsep{-1em}
    \item Riemannian Geometry*
\end{itemize}
}

%------------------------------------------------

\entries
{LOGIC AND THEORETICAL COMPUTING SCIENCE}
{
\begin{itemize}
    \setlength\itemsep{-1em}
    \item Mathematical Logic (B+)\\
    \item Nonstandard Logical Systems (A)\\
    \item Formal Language Theory (A)
\end{itemize}
}

%------------------------------------------------

\end{entrylist}

\pagebreak[2]
\subsection{Relevant Work \& Experience}

\begin{entrylist}

\entry
{2017}
{UNIVERSITY OF ALBERTA}
{Edmonton, Alberta}
{\emph{Undergraduate Research Assistant} \\
Worked with combinatorial optimization researcher Zachary Friggstadt to come up with novel techniques for approximation algorithms to variants of the capacitated vehicle routing problem. We used Lagrangian preserving approximations for linear programming relaxations of the problem to obtain solutions to vehicle routing problems with cardinality requirements.}

\entry
{2015}
{UNIVERSITY OF ALBERTA}
{Edmonton, Alberta}
{\emph{`Tangible Introduction To Computing Science' Teaching Assistant} \\
Advised students in the honours stream of Computing Science who were taking CMPUT 275, a class which introduced students to basic algorithmics, such as asymptotic analysis, divide and conquer, dynamic programming, and such. Led office hours weekly and marked assignments.}

%------------------------------------------------

\entry
{2013-Now\ \ \ \ \ }
{Competitive Programming club}
{}
{\emph{Competitor} \\
Strong Competitor in Competitive Programming, which presses competitors to find fast solutions to combinatorial problems. Won the Microsoft 2014 Coding for Cash competition, placed 4th in the Alberta Collegiate programming contest in 2014 and 2015. Coached by Zachary Friggstadt (\href{mailto:zacharyf@ualberta.ca}{zacharyf@ualberta.ca}), ACM world finalist.}

\end{entrylist}

\pagebreak[4]
\subsection{Summer Internships}

\begin{entrylist}

%------------------------------------------------

\entry
{2016}
{Microsoft}
{Redmond, Washington}
{\emph{Universal Store Mobile Device Forensics} \\
Developed algorithms for the mobile section of the Microsoft fraud detection team, which uses machine learning techniques on large data sets to predetermine fraud and protect the accounts of Microsoft store customers. The software I designed is set to be implemented on the two most popular Microsoft phone applications. }

%------------------------------------------------

\entry
{2015}
{Microsoft}
{Redmond, Washington}
{\emph{Universal Store Spell Correction} \\
Developed algorithms for data linkage. Utilizing various data-cleansing methods together with the Azure and Bing data-analysis packages, cleansed Microsoft's business partner database, removing redundant info, reducing database entries by 20\%. My manager for this project was Aman Kansal (\href{mailto:Kansal@microsoft.com}{Kansal@microsoft.com}). I also worked off-hours with a group of interns to send robot adventurers around the world (\href{http://www.projectatlas.ms/}{http://www.projectatlas.ms/}), and organized weekly talk sessions!}

%------------------------------------------------

\entry
{2014}
{University of Alberta}
{Edmonton, Alberta}
{\emph{Natural Language Processing and Cognate Identification} \\
Worked with the NLP group at the University of Alberta to develop cognate recognition algorithms. Successfully pushed to create a centralized database for storing cognate information, simplifying the learning process. This program was successfully used by linguists at the University of Alberta to understand the Totonac language group. Garrett Nicolai supervised the project (\href{mailto:Nicolai@ualberta.ca}{Nicolai@ualberta.ca}).}

%------------------------------------------------

\entry
{2013}
{University of Alberta}
{Edmonton, Alberta}
{\emph{Reinforcement Learning GAMES group} \\
Implemented efficient abstraction algorithms to create a Sokoban solver for the RLAI group at the University of Alberta, under mentor Harm Van Seijen (\href{mailto:Harm.Van.Seijen@gmail.com}{Harm.Van.Seijen@gmail.com}).}

%------------------------------------------------

\end{entrylist}

\newpage

%----------------------------------------------------------------------------------------
%	AWARDS SECTION
%----------------------------------------------------------------------------------------

\pagebreak[2]
\section{Awards}

\begin{entrylist}

\entry
{2017}
{Faculty of Science Graduate Award}
{Graduate Support Initiative}
{}

%%------------------------------------------------

\entry
{2017}
{NSERC Undergraduate Student Research Award}
{Alberta Scholarships}
{To Nurture the interest and fully develop potential for a research career in the natural sciences and engineering. Recieved twice, both in the spring and summer, but only accepted in the summer.}

\entry
{2014-2016}
{Jason Lang Scholarship}
{Alberta Scholarships}
{(3 Time Award Winner) Awarded to students Alberta post-secondary students continuing full-time in undergraduate programs with outstanding academic achievements.}

%------------------------------------------------

\entry
{2014}
{NSERC Undergraduate Student Research Award}
{Alberta Scholarships}
{To Nurture the interest and fully develop potential for a research career in the natural sciences and engineering.}

%------------------------------------------------

\entry
{2013}
{Academic Excellence Scholarship}
{University of Alberta}
{Awarded to students with superior academic achievement entering the first year of an undergraduate degree program at the University of Alberta.}

%%------------------------------------------------

\entry
{2013}
{Faculty of Science Academic Excellence Scholarship}
{University of Alberta}
{Awarded annually on the basis of superior academic achievement to students entering the first year of an undergraduate degree program in the Faculty of Science at the University of Alberta.}

%------------------------------------------------

\entry
{2013}
{Alexander Rutherford Achievement Scholarship}
{Alberta Scholarships}
{To recognize and reward academic achievement at the senior high school level and to encourage students to pursue post-secondary studies.}

%%------------------------------------------------

\end{entrylist}


%----------------------------------------------------------------------------------------
%   EDUCATION SECTION
%----------------------------------------------------------------------------------------

\section{Education}

\begin{entrylist}

%------------------------------------------------

\entry
{2017-Present\ \ \ \ \ \ \ }
{Masters {\normalfont in Mathematics}}
{The University of British Columbia}

%------------------------------------------------

\entry
{2013-2017\ \ \ \ \ \ \ }
{Bachelors {\normalfont in Computing Science}}
{The University of Alberta}

%------------------------------------------------

\entry
{2011-2013}
{International Baccalaureate {\normalfont High School Diploma}}
{Harry Ainlay High School}

%------------------------------------------------

\end{entrylist}

%----------------------------------------------------------------------------------------
%	COMMUNICATION SKILLS SECTION
%----------------------------------------------------------------------------------------

%----------------------------------------------------------------------------------------
%	INTERESTS SECTION
%----------------------------------------------------------------------------------------

%\section{Interests}

%----------------------------------------------------------------------------------------
%	PUBLICATIONS SECTION
%----------------------------------------------------------------------------------------

%\section{publications}

%\printbibsection{article}{article in peer-reviewed journal} % Print all articles from the bibliography

%\printbibsection{book}{books} % Print all books from the bibliography

%\begin{refsection} % This is a custom heading for those references marked as "inproceedings" but not containing "keyword=france"
%\nocite{*}
%\printbibliography[sorting=chronological, type=inproceedings, title={international peer-reviewed conferences/proceedings}, notkeyword={france}, heading=bibheading]
%\end{refsection}

%\begin{refsection} % This is a custom heading for those references marked as "inproceedings" and containing "keyword=france"
%\nocite{*}
%\printbibliography[sorting=chronological, type=inproceedings, title={local peer-reviewed conferences/proceedings}, keyword={france}, heading=bibheading]
%\end{refsection}

%\printbibsection{misc}{other publications} % Print all miscellaneous entries from the bibliography

%\printbibsection{report}{research reports} % Print all research reports from the bibliography

%----------------------------------------------------------------------------------------

\end{document}