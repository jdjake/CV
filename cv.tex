%%%%%%%%%%%%%%%%%%%%%%%%%%%%%%%%%%%%%%%%%
% cv-friggeri-x 1.0 (01/01/2016)
% XeLaTeX Template
%
% Based on:
% Friggeri Resume/CV
% Version 1.2 (3/5/15)
%
% Original author:
% Adrien Friggeri (adrien@friggeri.net)
% https://github.com/afriggeri/CV
% 
% Modified by:
% Nadorrano
% https://github.com/Nadorrano/cv-friggeri-x
%
% License:
% MIT (https://opensource.org/licenses/MIT)
% CC BY-NC-SA 3.0 (http://creativecommons.org/licenses/by-nc-sa/3.0/)
%
% Important notes:
% This template needs to be compiled with XeLaTeX and the bibliography,
% if used, needs to be compiled with biber rather than bibtex.
%
%%%%%%%%%%%%%%%%%%%%%%%%%%%%%%%%%%%%%%%%%

\documentclass[a4paper]{cv-friggeri}
% Add `a4paper` to set a4 paper size
% Add `nocolors` to remove colors from the document
% Add `lightheader` to change the dark background of the header to white

\usepackage{marvosym} % needed to print glyphs for email, cell phone etc.

\addbibresource{bibliography.bib} % Specify the bibliography file to include publications

\begin{document}

\header{Jacob\ }{Denson}{} % Your name and current job title/field

%----------------------------------------------------------------------------------------
%	SIDEBAR SECTION
%----------------------------------------------------------------------------------------

\begin{aside} % In the aside, each new line forces a line break
\section{Research Interests}
Harmonic Analysis, Geometric Measure Theory, and Nonlinear Wave Equations
~
\vspace{-1.3em}\section{Contact Information}
\href{mailto:jcdenson@wisc.edu}{jcdenson@wisc.edu}
%\href{https://github.com/jdjake}{{\bf Github Profile}: jdjake}
\href{https://math.stackexchange.com/users/120724/jacob-denson}{{\bf Math Stack Overflow:} jacob-denson}
\href{https://jdjake.github.io/}{{\bf Website:} https://jdjake.github.io/}
~
\vspace{-1.3em}\section{Education}
{\bf 2020-2025}
PhD in Mathematics\\
at the University of Wisconsin, Madison
\emph{Advisor}:\\Andreas Seeger
{\bf 2017-2019}
{MSc in Mathematics}\\
at the University of British Columbia
\emph{Advisors}:\\Malabika Pramanik\\Joshua Zahl
{\bf 2013-2017}
BSc in Computing Science at the University of Alberta.
~
\vspace{-1.3em}\section{Teaching Assistantships}
\emph{2024}
Multivariate Calculus
\vspace{0.2em}Linear Algebra
\vspace{0.5em}\emph{2023}
Preparatory Algebra Lecturer
\vspace{0.2em}Introduction to Discrete Mathematics
\vspace{0.5em}\emph{2022}
Elementary Matrix and Linear Algebra
\vspace{0.2em}Introduction to Partial Differential Equations
\vspace{0.2em}Introduction to Discrete Mathematics
%~
%\section{languages}
%english mother tongue
%spanish \& italian fluency
%\section{programming}
%{\color{red} $\varheartsuit$} \LaTeX, JavaScript
%Python, C++, PHP
%CSS3 \& HTML5
%~
%\section{skills}
%Machine Learning: \hspace{5mm}\null
%\grade{4.5} 
%Optimization:     \hspace{5mm}\null  
%\grade{4}
%Computer Vision:  \hspace{5mm}\null
%\grade{3}
%Text Mining:      \hspace{5mm}\null
%\grade{2.5}
%\section{Languages}
%English, Elementary Mandarin, Python, Perl, C++, C, C\#, Matlab, HTML, Javascript, Latex
%~
%\section{Graduate Awards}
%\emph{2023-2024}
%Henry Schaerf Mathematics Graduate Award
%\emph{2019}
%February Fourier Talks Poster Presentation Award (2nd Place)
%\emph{2018}
%NSERC CGSM Master's Scholarship
%UBC Science Graduate Award (2nd Time)
%UBC Science Graduate Award
\end{aside}

%----------------------------------------------------------------------------------------
%	EDUCATION SECTION
%----------------------------------------------------------------------------------------

%\section{Research Projects}

%\begin{entrylist}

%\entry
%{2021-Now}
%{Radial Fourier Multipliers on Manifolds}
%{}
%{\emph{Collaboration with Dr. Andreas Seeger}. The goal of this project is to use the technology of Fourier Integral Operators to extend results about radial Fourier multipliers on Euclidean space to obtain bounds for multipliers for the Laplace-Beltrami operator on compact Riemannian manifolds.}

%Our current progress is detailed in notes linked \href{https://github.com/jdjake/Notes/blob/master/Research/FourierIntegralOperators/RadialMultipliers.pdf}{here}.

%------------------------------------------------

%\entry
%{2020-Now}
%{Salem Sets Avoiding Patterns}
%{}
%{The goal of this project is to explore relations between the Fourier transform and methods of high dimensional probability. Using these methods, I constructed large Salem sets avoiding patterns.

%new tools which enable one to extend pattern avoidance methods which construct sets with large Hausdorff dimension avoiding patterns, to construct large Salem sets avoiding patterns. I am currently exploring ways to improve the probabilistic construction methods discussed in the paper ``Large Salem Sets Avoiding Nonlinear Configurations'' to construct optimal Salem sets avoiding patterns, as well as exploring the use of techniques based in extremizer theory to calculate the Fourier dimension of surfaces in Euclidean space with large codimension.}

%------------------------------------------------

%\entry
%{2018-2019}
%{Large Sets Avoiding Rough Patterns}
%{}
%{\emph{Collaboration with Dr. Malabika Pramanik and Dr. Joshua Zahl.} In this project, we found subsets of Euclidean space with large Hausdorff dimension avoiding point configurations.}%, which might be described as having a `rough' character, such as those related to additive structure on fractals.}

%------------------------------------------------

%\entry
%{2017}
%{Lagrangian Preserving Approximation for Vehicle Routing}
%{}
%{\emph{Collaboration with Dr. Zachary Friggstad}. This project involves using the Lagrangian preserving approximation technique combined with a novel linear relaxation of variants of the vehicle routing problem to obtain state of the art approximation algorithms. Our work is detailed in notes linked \href{https://github.com/jdjake/Notes/raw/master/Research/VehicleRouting/VehicleRouting.pdf}{here}.}

%-------------------------------------------------

%\entry
%{2015-2016}
%{Universal Store Record Linkage}
%{}
%{\emph{Collaboration with Dr. Aman Kansal and Ram Chandrasekaran}. As an intern, developed data linkage methods in Microsoft's Universal Store department in Redmond, Washington. My main responsibility was reading articles and white papers on the record linkage problem, and developing the ideas in those papers into usable software. My software removed redundant information from Microsoft's database, which took up 20\% of the size of the entire database.}

%-------------------------------------------------

%\entry
%{2014}
%{Cognate Identification}
%{}
%{\emph{Collaboration with Garret Nicolai and Dr. Greg Kondrak}. Developed cognate recognition algorithms with the NLP group at the University of Alberta. Created a centralized database for storing cognate information. This program was successfully used by linguists at the University of Alberta to understand the Totonac language group. Garrett Nicolai supervised the project (Nicolai@ualberta.ca).}

%\end{entrylist}

%----------------------------------------------------------------------------------------
% PUBLICATIONS
%----------------------------------------------------------------------------------------

\section{Publications}

\cite{characterizationpaper}
\cite{salemsetspaper}
\cite{mscthesis}
\cite{largesetsroughpatterns}
\cite{notesfromthemargin}

%\printbibsection{article}{Articles} % Print all articles from the bibliography

%\printbibsection{book}{books} % Print all books from the bibliography

%\begin{refsection} % This is a custom heading for those references marked as "inproceedings" but not containing "keyword=france"
%\nocite{*}
%\printbibliography[sorting=chronological, type=inproceedings, title={international peer-reviewed conferences/proceedings}, notkeyword={france}, heading=bibheading]
%\end{refsection}

%\begin{refsection} % This is a custom heading for those references marked as "inproceedings" and containing "keyword=france"
%\nocite{*}
%\printbibliography[sorting=chronological, type=inproceedings, title={local peer-reviewed conferences/proceedings}, keyword={france}, heading=bibheading]
%\end{refsection}

%\printbibsection{misc}{other publications} % Print all miscellaneous entries from the bibliography

%\printbibsection{report}{research reports} % Print all research reports from the bibliography

%----------------------------------------------------------------------------------------

\vspace{-1.5em} \section{Graduate Awards}

\begin{entrylist}

\entry
{2024}
{The Excellence in Research Graduate Student Award}
{}
{Awarded to three mathematics graduate students each year for significant and substantial contributions to research mathematics, as part of their thesis work towards a Ph.D.}

\entry
{2023}
{Henry Schaerf Mathematics Graduate Award}
{}
{Awarded to three mathematics graduate students each year who have demonstrated promise in their academic work.}

\entry
{2019}
{February Fourier Talks Poster Presentation Award (2nd Place)}
{}
{Awarded to the top three posters presented at the February Fourier Talks poster session. The poster, entitled ``Fractals Avoiding Fractal Configurations'', can be found at \href{https://jdjake.github.io/notes}{https://jdjake.github.io/notes}.}

\entry
{2018}
{NSERC CGSM Master's Scholarship}
{}
{Awarded to 3000 students in Master's programs across all disciplines in the sciences and engineering across Canada, providing financial support to high-calibre scholars.}% Provides financial support to high-calibre scholars allowing them to concentrate more fully on their studies in their chosen fields.}

\end{entrylist}

%\emph{2023-2024}
%Henry Schaerf Mathematics Graduate Award
%\emph{2019}
%February Fourier Talks Poster Presentation Award (2nd Place)
%\emph{2018}
%NSERC CGSM Master's Scholarship
%UBC Science Graduate Award (2nd Time)
%UBC Science Graduate Award

%\newpage

%----------------------------------------------------------------------------------------
% CONFERENCE PRESENTATIONS
%----------------------------------------------------------------------------------------

\vspace{-2em} \section{Conference Presentations}

\begin{entrylist}

\entry
{2024-Now}
{Characterizations of Boundedness For Multipliers of Spherical Harmonics}
{}
{\emph{Presented at}:
%
\begin{itemize}
	\item \emph{The 2024 Ohio River Analysis Meeting.}
	\item \emph{Poster at the 2024 Madison Lectures in Harmonics Analysis}
	\item \emph{The AMS 2024 Spring Central Sectional Meeting}% Special Session on Harmonic Analysis and Incidence Geometry}.	
	\item \emph{The AMS 2024 Fall Western Sectional Meeting}% Special Session on Harmonic Analysis, Partial Differential Equations, and Spectral Theory}
	\item \emph{The CMS 2024 Winter Meeting}
\end{itemize}
%
%A talk discussing my work characterizing multipliers of spherical harmonic expansions whose dilates are uniformly bounded on $L^p$.}
}

\entry
{2020-2021}
{Salem Sets Avoiding Patterns}
{}
{\emph{Presented at}:
%
\begin{itemize}
	\item \emph{The 2020 Ohio River Analysis Meeting}.
	\item \emph{The University of Wisconsin Analysis Student Seminar}.
	\item \emph{The 17th Prairie Analysis Seminar}.
\end{itemize}
%
%A talk discussing my work constructing high dimensional Salem sets avoiding configurations.}% I emphasized the square root cancellation phenomena necessary to extend previous results on Hausdorff dimension to the setting of Fourier dimension, and how this can be obtained using the theory of concentration of measure from probability theory.}
}

\entry
{2018-2019}
{Fractals Avoiding Fractal Sets}
{}
{\emph{Presented at}:
%
\begin{itemize}
	\item \emph{The 2018 Mid-Atlantic Analysis Meeting}.
	\item \emph{The 2018 CMS Winter Meeting}.
	\item \emph{The 2019 Geometric and Harmonic Analysis (GAHA) Conference}.
	\item \emph{Poster at the 2019 February Fourier Talks. Awarded Prize for 2nd Best Poster out of 19 participants}.
	\item \emph{Poster at the 2019 Madison Lectures in Fourier Analysis}.
\end{itemize}
%
%A talk discussing my work with Dr. Malabika Pramanik and Dr. Joshua Zahl on constructing high dimensional sets avoiding configurations. I emphasized the idea behind the discretization of a problem when working a single scale.}
}

\end{entrylist}

\section{Select Expository Talks}

\emph{Talk Notes can be found at my website: \href{https://jdjake.github.io/notes}{https://jdjake.github.io/notes}:}
\\

\begin{entrylist}

\entry
{2024}
{\href{https://github.com/jdjake/Notes/raw/master/Talks/TalkNotes/DensonRaiChoudhuri.pdf}{On Trilinear Oscillatory Integral Inequalities and Related Topics (After Christ)}}
{}
{UW Madison 2022 Spring School: Multilinear Singular and Oscillatory Integrals with Applications}

\entry
{2023}
{\href{https://github.com/jdjake/Notes/raw/master/Talks/TalkNotes/DecouplingEntangledParaproducts.pdf}{Detangling Entangled Paraproducts (After Durcik)}}
{}
{Bonn 2023 Summer School: Analysis of Multiple Ergodic Averages}

\entry
{2023}
{\href{https://github.com/jdjake/Notes/raw/master/Talks/TalkNotes/LaxParametrix.pdf}{The Lax H\"{o}rmander Parametrix for the Half-Wave Equation}}
{}
{UW Madison Graduate Analysis Seminar}

\entry
{2022}
{\href{https://github.com/jdjake/Notes/raw/master/Talks/TalkNotes/AnticoncentrationPolynomialDecompositions.pdf}{Anticoncentration and Polynomial Decompositions (After Kane)}}
{}
{UC Irvine 2022 Summer School: Learning Theory and Fourier Analysis}

\entry
{2022}
{\href{https://github.com/jdjake/Notes/raw/master/Talks/TalkNotes/NodalGeometryAndBrownianMotion.pdf}{Nodal Domains on Riemannian Manifolds via Diffusion Processes (After Steinerberger)}}
{}
{Bonn 2022 Summer School: Nodal Domains and Landscape Functions}

\entry
{2022}
{\href{https://github.com/jdjake/Notes/raw/master/Talks/TalkNotes/LogarithmicImprovementsSpectralBands.pdf}{Logarithmic Improvements to $L^p$ Bounds for Laplace-Beltrami Eigenfunctions on Hyperbolic Manifolds (After Sogge and Blair)}}
{}
{UW Madison 2022 Summer School: Harmonic Analysis on Manifolds}

\entry
{2021}
{\href{https://github.com/jdjake/Notes/raw/master/Talks/TalkNotes/RankDecreasingAndCapacity.pdf}{Algorithmic Aspects of Brascamp Lieb Inequalities (After Garg, Gurvits, Oliveira, and Wigderson)}}
{}
{Bonn 2021 Summer School: Brascamp Lieb Inequalities}

\entry
{2019}
{\href{https://github.com/jdjake/Notes/raw/master/Talks/TalkNotes/SzemerediTrotterin3D.pdf}{Incidence Theorems over Field of Arbitrary Characteristic (after Koll\'{a}r)}}
{}
{Math 616A Class: The Polynomial Method}

\entry
{2018}
{\href{https://github.com/jdjake/Notes/raw/master/Talks/TalkNotes/HodgeTheory.pdf}{Hodge Theory: Harmonic Analysis in Topology}}
{}
{Math 529 Class: Differential Topology}

%\entry
%{2018}
%{\href{https://github.com/jdjake/Notes/raw/master/Talks/TalkNotes/ThetaFunctions.pdf}{Theta Functions}}
%{}
%{Math 600D Class: Modular Forms}

%\entry
%{2018}
%{\href{https://github.com/jdjake/Notes/raw/master/Talks/TalkNotes/RadonTransformsAndExceptionalProjections.pdf}{Radon Transforms and Exceptional Projections}}
%{}
%{Math 542 Class: Geometric Measure Theory}

\entry
{2017}
{\href{https://github.com/jdjake/Notes/raw/master/Talks/TalkNotes/ProofsInThreeBits.pdf}{Proofs in Three Bits or Less}}
{}
{UBC Graduate Seminar}

%\entry
%{2016}
%{\href{https://github.com/jdjake/Notes/raw/master/Talks/TalkNotes/FourierStieltjes.pdf}{Why Physicists Care About the Fourier-Stieltjes Transform}}
%{}
%{Math 642 Class: Non Commutative Harmonic Analysis}

%\entry
%{2016}
%{\href{https://github.com/jdjake/Notes/raw/master/Talks/TalkNotes/PontrayaginDuality.pdf}{A Brief Respite in Abelian Harmonic Analysis}}
%{}
%{Math 642 Class: Non Commutative Harmonic Analysis}

\end{entrylist}

\begin{asidenotit}
%\section{Graduate Awards (Continued)}
%\emph{2017}
%U of A Dean's Silver Medal in Science
%NSERC USRA
%(2nd and 3rd Time)
%\emph{2016}
%Jason Lang Scholarship
%(3rd Time)
%\emph{2015}
%Jason Lang Scholarship
%(2nd Time)
%\emph{2014}
%NSERC USRA
%Jason Lang Scholarship
%\emph{2013}
%U of A Academic Excellence Scholarship
%U of A Science Academic Excellence Scholarship
%Alexander Rutherford Achievement Scholarship
%~
\section{Teaching Assistantships (Continued)}
\emph{2021}
Algebra \& Trigonometry
\vspace{0.2em}Calculus
\vspace{0.5em} \emph{2020}
Calculus with Algebra \& Trigonometry
\vspace{0.5em} \emph{2019}
Multivariate Calculus
\vspace{0.2em}Graph Theory
\vspace{0.5em} \emph{2018}
Introduction to Discrete Mathematics
\vspace{0.2em}Introduction to Probability
\vspace{0.5em} \emph{2017}
Calculus for Forestry Students
\vspace{0.2em}Calculus for Business Students
\vspace{0.5em} \emph{2015}
Tangible Introduction to Computer Science Undergraduate TA
~
\vspace{-1.3em}\section{Service}
\emph{2024}
UW Madison Fall 2024 Directed Reading Program Mentor \vspace{0.5em}
UW Madison Spring 2024 Directed Reading Program Mentor \vspace{0.5em}
Mega Math Meet Problem Committee Member
\end{asidenotit}

\end{document}