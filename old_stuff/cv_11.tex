%%%%%%%%%%%%%%%%%%%%%%%%%%%%%%%%%%%%%%%%%
% Friggeri Resume/CV
% XeLaTeX Template
% Version 1.2 (3/5/15)
%
% This template has been downloaded from:
% http://www.LaTeXTemplates.com
%
% Original author:
% Adrien Friggeri (adrien@friggeri.net)
% https://github.com/afriggeri/CV
%
% License:
% CC BY-NC-SA 3.0 (http://creativecommons.org/licenses/by-nc-sa/3.0/)
%
% Important notes:
% This template needs to be compiled with XeLaTeX and the bibliography, if used,
% needs to be compiled with biber rather than bibtex.
%
%%%%%%%%%%%%%%%%%%%%%%%%%%%%%%%%%%%%%%%%%

% Add 'print' as an option into the square bracket to remove colors from this template for printing
\documentclass{friggeri-cv}

% Specify the bibliography file to include 
% \addbibresource{bibliography.bib}

\newenvironment{changemargin}[2]{%
\begin{list}{}{%
\setlength{\topsep}{0pt}%
\setlength{\leftmargin}{#1}%
\setlength{\rightmargin}{#2}%
\setlength{\listparindent}{\parindent}%
\setlength{\itemindent}{\parindent}%
\setlength{\parsep}{\parskip}%
}%
\item[]}{\end{list}}

\begin{document}

\header{Jacob}{Denson}{Mathematics Student} % Your name and current job title/field

%----------------------------------------------------------------------------------------
%	SIDEBAR SECTION
%----------------------------------------------------------------------------------------

\begin{aside} % In the aside, each new line forces a line break
\section{Research Interests}
Harmonic Analysis, Geometric Analysis, Additive Combinatorics.
\section{Contact Information}
\href{mailto:denson@math.ubc.ca}{denson@math.ubc.ca} \vspaces
\href{https://github.com/jdjake}{{\bf Github Profile}: jdjake} \vspaces
\href{http://stackoverflow.com/users/2601483/jacob-denson}{{\bf Stack Overflow:} jacob-denson} \vspaces
\href{https://jdjake.github.io/}{{\bf Website:} https://jdjake.github.io/}
\section{Languages}
English, Elementary German, Very Rusty Elementary Chinese, Python, Perl, C++, C, C\#, Matlab, HTML, Javascript, Latex
(This resume is proof!)
\section{Education}
\subsection{2017-Present}
Master's in Mathematics
University of British Columbia.
\subsection{2013-2017}
Bachelors in Computing Science
University of Alberta.
\end{aside}

%----------------------------------------------------------------------------------------
%   KEY SKILLS
%----------------------------------------------------------------------------------------

%\begin{changemargin}{+3.5cm}{0cm}

\section{Summary}

I am a masters student at the University of British Columbia, applying my strong and diverse foundation in mathematical knowledge to do research in the harmonic analysis research group, studying continuous variants of discrete configuration avoidance problems emerging from additive combinatorics. My previous work in theoretical computing science has given me a strong knowledge of the algorithmic viewpoint of problems, which gives me a fresh perspective on classical ideas in the field. I am currently working on the problem of finding high dimensional fractals avoiding patterns.

%----------------------------------------------------------------------------------------
%	WORK EXPERIENCE SECTION
%----------------------------------------------------------------------------------------

\pagebreak[3]
\section{Talks\footnote{Notes for my talks can be found on my website: \href{https://jdjake.github.io/}{https://jdjake.github.io/}}} % \& Publications}

\begin{entrylist}

%------------------------------------------------

\entry
{2018}
{CMS WINTER MEETING \& MAAM 2018}
{CMS \& University of Virginia}
{\emph{Fractals Avoiding Fractal Sets}\\
    A twenty minute talk discussing my solution to a research problem constructing fractal sets of high Hausdorff dimension avoiding patterns. I emphasized the idea behind the switch from a continuous problem to a discrete scale argument, as well as discussing the strategy of the hypergraph avoidance method at the single scale.
}

%------------------------------------------------

\entry
{2018}
{DIFFERENTIAL TOPOLOGY CLASS}
{University of British Columbia}
{\emph{Hodge Theory: Harmonic Analysis in Topology}\\
    An hour talk discussing how the eigenfunctions of the Laplacian on a Riemannian manifold reflect the topology of the underlying manifold. I introduced the inner product of differential forms, the Hodge star, and the Laplace-Beltrami operator, and how these eigenfunctions can be used to give almost trivial proofs of major results about De Rham cohomology.
}

%------------------------------------------------

\entry
{2018}
{MODULAR FORMS CLASS}
{University of British Columbia}
{\emph{Theta Functions}\\
    An hour long talk discussing how the theory of theta functions fits in with the general theory of modular forms once we introduce half weight forms and a modular symmetry with respect to a Dirichlet character. Using this theory, we prove Fermat's theorem on the sums of two squares, and Jacobi's theorem on the sums of four squares.
}

%------------------------------------------------

\entry
{2018}
{TOPICS IN HARMONIC ANALYSIS CLASS}
{University of British Columbia}
{\emph{Radon Transform and Exceptional Projections}\\
    An hour talk connecting the Marstrand projection problem in geometric measure theory to harmonic analysis using the Radon transform. Bounding variants of the Radon transform gives results about the dimension of the set of projections where Marstrand's theorem fails. Based on Daniel M. Oberlin's paper ``Restricted Radon Transforms and Projections of Planar Sets''.
} 

\end{entrylist}
%\end{changemargin}

\newpage

\begin{asidenotit}
\section{Awards}
\subsection{2018}
NSERC CGSM \vspaces
UBC Science Graduate Award
(2nd Time)
\subsection{2017}
UBC Science Graduate Award \vspaces
U of A Dean's Silver Medal in Science \vspaces
NSERC USRA
(2nd and 3rd Time)
\subsection{2016}
Jason Lang Scholarship
(3rd Time)
\subsection{2015}
Jason Lang Scholarship
(2nd Time)
\subsection{2014}
NSERC USRA \vspaces
Jason Lang Scholarship
\subsection{2013}
U of A Academic Excellence Scholarship \vspaces
U of A Science Academic Excellence Scholarship \vspaces
Alexander Rutherford Achievement Scholarship
\end{asidenotit}

\begin{entrylist}

%------------------------------------------------

\entry
{2017}
{GRADUATE SEMINAR}
{University of British Columbia}
{\emph{Proofs in Three Bits or Less}\\
An hour talk introducing nonspecialists to the theory of probabilistically checkable proofs, and PCP theory. By changing the language by which we discuss the theory from accessing random bits from a string, to `playing a game of 20 questions', I introduced a novel way to discuss the theory which avoids the technicalities of the field, making the talk accessible to students without any background in theoretical computing science. The ideas behind this talk were the basis for my published article in the 2018 edition of the Notes from the Margin expository journal.}

%------------------------------------------------

\entry
{2016}
{NONCOMMUTATIVE HARMONIC ANALYSIS CLASS}
{University of Alberta, Canada}
{\emph{Why Physicists Care About The Fourier-Stieltjes Transform}\\
A 20 minute talk emphasizing the naturality of the generalization of the Fourier transform to the Fourier-Stieltjes transform by proving the weak $*$ density of $L^1(G)$ in $M(G)$, and discussing why this matters.}

\entry
{2016}
{NONCOMMUTATIVE HARMONIC ANALYSIS CLASS}
{University of Alberta, Canada}
{\emph{A Brief Respite In Abelian Analysis}\\
A 20 minute talk introducing the abstract Fourier transform on abelian locally compact groups, and discussing the generalization of the Poincare summation formula to this domain, which hints at the depth of Pontrayagin duality.}

\end{entrylist}

\begin{entrylist}

%------------------------------------------------

\entry
{2016}
{CUMC CONFERENCE}
{University of Victoria, Vancouver Island}
{\emph{On Molecular Gases and the Natural Numbers}\\
A talk introducing Ergodic theory to undergraduate students, and emphasizing its relation to a variety of problems in mathematics, especially number theory.}

%------------------------------------------------

\entry
{2016}
{ALGEBRAIC TOPOLOGY GRAUATE CLASS}
{University of Alberta, Canada}
{\emph{Vector Fields, Hex, and Jordan Curves}\\
A 20 minute talk on the Brouwer fixed-point theorem, emphasizing the intuitive vector field interpretation of the theorem, and discussing how the fixed-point theorem relates to the combinatorial game of hex, reflecting the nice interweaving of discrete and point-set methods in algebraic topology.}

%\newgeometry{margin=1in,top=1cm,right=1cm,bottom=1cm}

%------------------------------------------------
    
\entry
{2015}
{MICROSOFT INTERN TALKS}
{Microsoft Campus, Redmond}
{\emph{Category Theory for Computer Programmers}\\
My original talk on category theory, shortened to a 20 minutes talk, and edited to reduce mathematical prerequisites and to emphasize the practical uses for the average programmer, as a talk in the weekly talk seminar for inerns I ran about various interesting topics in computing science.}

%------------------------------------------------

\entry
{2015}
{HONOURS COMPUTING SCIENCE SEMINAR}
{University of Alberta}
{\emph{Category Theory and its relation to Computing Science}\\
an hour-long talk introducing the subject to Honours computing scientists and emphasizing its relation to the Curry Howard isomorphism.}

%------------------------------------------------

\entry
{2014}
{NLP RESEARCH GROUP}
{University of Alberta}
{\emph{Cognates for Reconstruction of Native American Language groups}\\
a 20 minute talk emphasizing my work over the summer and explaining the organization method and SVM classification method for identifying cognates.}

%------------------------------------------------

\entry
{2013}
{RLAI TEA TIME TALKS}
{University of Alberta}
{\emph{Room Abstraction in Sokoban}\\
a 15 minute talk introducing the game of Sokoban, its combinatorial issues, and room abstraction as an aid to attacking the game.}

\end{entrylist}

\section{Experience}

\subsection{Selected Mathematical Knowledge (Including Textbooks Read)}

\begin{entrylist}

%------------------------------------------------

\entries
{FUNCTIONAL ANALYSIS}
{
\begin{itemize}
    \setlength\itemsep{-1em}
    \item Banach Spaces (Conway, Lax)\\
    \item Weak Topologies \& Distribution Theory (Rudin)\\
    \item Operator Algebras (Kadison \& Ringrose)
\end{itemize}
}

%------------------------------------------------

%------------------------------------------------

\entries
{COMPLEX ANALYSIS}
{
\begin{itemize}
    \setlength\itemsep{-1em}
    \item Complex Variables (Ahlfors, Stein \& Sharkarchi Vol. 2)\\
    \item Modular Forms (Milne, Diamond \& Shurman)\\
    \item Riemann Surfaces (Gunning)
\end{itemize}
}

%------------------------------------------------

\entries
{HARMONIC ANALYSIS}
{
\begin{itemize}
    \setlength\itemsep{-1em}
    \item Euclidean Harmonic Analysis (Stein \& Sharkarchi Vol. 1, K\"{o}rner, Stein \& Weiss)\\
    \item Abstract Harmonic Analysis (Folland, Rudin, Hewit \& Ross)\\
    \item Partial Differential Equations (Evans)\\
    \item Applications to Geometric Measure Theory (Mattila: Fourier Anal\dots)
\end{itemize}
}

%------------------------------------------------

\entries
{ALGEBRA}
{
\begin{itemize}
    \setlength\itemsep{-1em}
    \item Galois Theory (Stewart, Lang: Algebra Chapters 4-6)\\
    \item Lie Algebras (Erdmann \& Wildon, Hall, Fulton \& Harris: Part 4)\\
    \item K Theory (Milnor, Adams)
\end{itemize}
}

%------------------------------------------------

\entries
{TOPOLOGY}
{
\begin{itemize}
    \setlength\itemsep{-1em}
    \item Topology (Munkres)\\
    \item Algebraic Topology (Hatcher)\\
    \item Differential Topology (Bott \& Tu, Tu: Diff. Geometry and Characteristic Classes)
\end{itemize}
}

\entries
{DISCRETE MATHEMATICS}
{
\begin{itemize}
    \setlength\itemsep{-1em}
    \item Combinatorial Optimization (Korte \& Vygen)\\
    \item Fourier Analysis of Boolean Functions (O'Donnell)\\
    \item Analytic Number Theory (Montgomery \& Vaughn Vol. 1)
\end{itemize}
}

\entries
{PROBABILITY THEORY}
{
\begin{itemize}
    \setlength\itemsep{-1em}
    \item Stochastic Processes (Lawler)\\
    \item Machine Learning (Hastie \& Tibshirani)\\
    \item Reinforcement Learning (Sutton \& Barto, Szepesvari)\\
    \item Multi Armed Bandits (Szepesvari \& Lattimore)\\
    \item Brownian Motion and Stochastic Integration (Rogers \& Williams Vol. 1)
\end{itemize}
}

%------------------------------------------------

\entries
{GEOMETRY}
{
\begin{itemize}
    \setlength\itemsep{-1em}
    \item Riemannian Geometry (Lee: Riemannannian Man. and an Intro. to Curvature)\\
    \item Algebraic Geometry (Fulton: Algebraic Curves, Harris: A First Course)\\
    \item Projective Geometry (Richter-Gebert)
\end{itemize}
}

%------------------------------------------------

\entries
{LOGIC AND THEORETICAL COMPUTING SCIENCE}
{
\begin{itemize}
    \setlength\itemsep{-1em}
    \item Mathematical Logic (Mendelson)\\
    \item Nonstandard Logic (Bimbo: Proof Theory \& Generalized Galois Logics)\\
    \item Algorithms (Cormer \& Leiserson \& Rivest \& Stein)\\
    \item Non Procedural Models of Computation (Hindley \& Selden)\\
    \item Computability Theory (Sipser, Arora \& Barak)
\end{itemize}
}

%------------------------------------------------

\end{entrylist}

\pagebreak[4]
\subsection{Research Projects}

\begin{entrylist}

\entry
{2017-2019}
{UNIVERSITY OF BRITISH COLUMBIA}
{Vancouver, Canada}
{\emph{Masters Research Student} \\
Worked with methods of geometric measure theory with Malabika Pramanik and Joshua Zahl in the harmonic analysis group at the University of British Columbia. Here we came up with novel techniques for constructing high dimensional fractals avoiding patterns. By employing a Cantor set type construction method, we were able to reduce the problem of avoiding patterns to the discrete problem of finding certain specialized independent sets in hypergraphs.
}

\entry
{2017}
{UNIVERSITY OF ALBERTA}
{Edmonton, Alberta}
{\emph{Undergraduate Research Assistant} \\
Worked with combinatorial optimization researcher Zachary Friggstadt to come up with novel techniques for approximation algorithms to variants of the capacitated vehicle routing problem. We used Lagrangian preserving approximations for linear programming relaxations of the problem to obtain solutions to vehicle routing problems with cardinality requirements.}

\entry
{2014}
{UNIVERSITY OF ALBERTA}
{Edmonton, Alberta}
{\emph{Natural Language Processing and Cognate Identification} \\
Worked with the NLP group at the University of Alberta to develop cognate recognition algorithms. Successfully pushed to create a centralized database for storing cognate information, simplifying the learning process. This program was successfully used by linguists at the University of Alberta to understand the Totonac language group. Garrett Nicolai supervised the project (\href{mailto:Nicolai@ualberta.ca}{Nicolai@ualberta.ca}).}

\entry
{2013}
{UNIVERSITY OF ALBERTA}
{Edmonton, Alberta}
{\emph{Reinforcement Learning GAMES group} \\
Implemented efficient abstraction algorithms to create a Sokoban solver for the RLAI group at the University of Alberta, under mentor Harm Van Seijen (\href{mailto:Harm.Van.Seijen@gmail.com}{Harm.Van.Seijen@gmail.com}).}

%------------------------------------------------

\end{entrylist}

%\pagebreak[4]
\subsection{Summer Internships}

\begin{entrylist}

%------------------------------------------------

\entry
{2016}
{MICROSOFT}
{Redmond, Washington}
{\emph{Universal Store Mobile Device Forensics} \\
Developed algorithms for the mobile section of the Microsoft fraud detection team, which uses machine learning techniques on large data sets to predetermine fraud and protect the accounts of Microsoft store customers. The software I designed is set to be implemented on the two most popular Microsoft phone applications. }

%------------------------------------------------

\entry
{2015}
{MICROSOFT}
{Redmond, Washington}
{\emph{Universal Store Spell Correction} \\
Developed algorithms for data linkage. Utilizing various data-cleansing methods together with the Azure and Bing data-analysis packages, cleansed Microsoft's business partner database, removing redundant info, reducing database entries by 20\%. My manager for this project was Aman Kansal (\href{mailto:Kansal@microsoft.com}{Kansal@microsoft.com}). I also worked off-hours with a group of interns to send robot adventurers around the world (\href{http://www.projectatlas.ms/}{http://www.projectatlas.ms/}), and organized weekly talk sessions!}

\end{entrylist}

\pagebreak[4]

\subsection{Teaching Assistantships}

\begin{entrylist}

\entry
{2018}
{UNVERSITY OF BRITISH COLUMBIA}
{Vancouver, Canada}
{\emph{`Introduction to Probability' Teaching Assistant} \\
Lead office hours and marked assignments on a first year course in probability.
}

\entry
{2018}
{UNIVERSITY OF BRITISH COLUMBIA}
{Vancouver, Canada}
{\emph{`Introduction to Discrete Mathematics' Teaching Assitant} \\
Advised students while leading office hours and marking assigments on basic combinatorics, including basic counting methods, asymptotics, graph theory, and generating function techniques.}

\entry
{2017}
{UNIVERSITY OF BRITISH COLUMBIA}
{Vancouver, Canada}
{\emph{Teaching Assitant to Two Calculus Courses} \\
Lead Workshops helping and testing students on the basic concepts of first semester calculus. Helped marked weekly assignments, midterms, and finals.
}

\entry
{2015}
{UNIVERSITY OF ALBERTA}
{Edmonton, Alberta}
{\emph{`Tangible Introduction To Computing Science' Teaching Assistant} \\
Advised students in the honours stream of Computing Science who were taking CMPUT 275, a class which introduced students to basic algorithmics, such as asymptotic analysis, divide and conquer, dynamic programming, and such. Led office hours weekly and marked assignments.}

\end{entrylist}

%\begin{entrylist}

%------------------------------------------------

%\entry
%{2013-2017\ \ \ \ \ }
%{Competitive Programming club}
%{}
%{\emph{Competitor} \\
%Strong Competitor in Competitive Programming, which presses competitors to find fast solutions to combinatorial problems. Won the Microsoft 2014 Coding for Cash competition, placed 4th in the Alberta Collegiate programming contest in 2014 and 2015. Coached by Zachary Friggstadt (\href{mailto:zacharyf@ualberta.ca}{zacharyf@ualberta.ca}), ACM world finalist.}

%------------------------------------------------

%\end{entrylist}

%----------------------------------------------------------------------------------------
%	AWARDS SECTION
%----------------------------------------------------------------------------------------

%\pagebreak[4]
%\section{Awards}

%\begin{entrylist}

%\entry
%{2018}
%{Canada Graduate Scholarship Master's Award}
%{National Science and Engineering Research Council}
%{To help develop research skills and assist in the training of highly qualified personnel by supporting students who demonstrate a high standard of achievement in undergraduate and early graduate studies.}

%\entry
%{2017-2018}
%{Faculty of Science Graduate Award}
%{Graduate Support Initiative}
%{(Two Time Award Winner) In recognition of academic achievement. Offered to support full-time study and/or research leading to a higher degree at The University of British Columbia}

%\entry
%{2017}
%{Dean's Silver Medal in Science}
%{University of Alberta}
%{To convocating students with superior academic achievement enrolled in an Honors program in the Faculty of Science at the University of Alberta.}

%%------------------------------------------------

%\entry
%{2017}
%{NSERC Undergraduate Student Research Award}
%{Alberta Scholarships}
%{(Awarded Twice in 2017, Accepted Once) To nurture the interest and fully develop potential for a research career in the natural sciences and engineering. Recieved twice, both in the spring and summer, but only accepted in the summer.}

%\entry
%{2014-2016}
%{Jason Lang Scholarship}
%{Alberta Scholarships}
%{(3 Time Award Winner) Awarded to students Alberta post-secondary students continuing full-time in undergraduate programs with outstanding academic achievements.}

%------------------------------------------------

%\entry
%{2014}
%{NSERC Undergraduate Student Research Award}
%{Alberta Scholarships}
%{To Nurture the interest and fully develop potential for a research career in the natural sciences and engineering.}

%------------------------------------------------

%\entry
%{2013}
%{Academic Excellence Scholarship}
%{University of Alberta}
%{Awarded to students with superior academic achievement entering the first year of an undergraduate degree program at the University of Alberta.}

%%------------------------------------------------

%\entry
%{2013}
%{Faculty of Science Academic Excellence Scholarship}
%{University of Alberta}
%{Awarded annually on the basis of superior academic achievement to students entering the first year of an undergraduate degree program in the Faculty of Science at the University of Alberta.}

%------------------------------------------------

%\entry
%{2013}
%{Alexander Rutherford Achievement Scholarship}
%{Alberta Scholarships}
%{To recognize and reward academic achievement at the senior high school level and to encourage students to pursue post-secondary studies.}

%%------------------------------------------------

%\end{entrylist}


%----------------------------------------------------------------------------------------
%   EDUCATION SECTION
%----------------------------------------------------------------------------------------

%\section{Education}

%\begin{entrylist}

%------------------------------------------------

%\entry
%{2017-2019\ \ \ \ \ \ \ }
%{Masters {\normalfont in Mathematics}}
%{The University of British Columbia}

%------------------------------------------------

%\entry
%{2013-2017\ \ \ \ \ \ \ }
%{Bachelors {\normalfont in Computing Science}}
%{The University of Alberta}

%------------------------------------------------

%\entry
%{2011-2013}
%{International Baccalaureate {\normalfont High School Diploma}}
%{Harry Ainlay High School}

%------------------------------------------------

%\end{entrylist}

%----------------------------------------------------------------------------------------
%	COMMUNICATION SKILLS SECTION
%----------------------------------------------------------------------------------------

%----------------------------------------------------------------------------------------
%	INTERESTS SECTION
%----------------------------------------------------------------------------------------

%\section{Interests}

%----------------------------------------------------------------------------------------
%	PUBLICATIONS SECTION
%----------------------------------------------------------------------------------------

%\section{publications}

%\printbibsection{article}{article in peer-reviewed journal} % Print all articles from the bibliography

%\printbibsection{book}{books} % Print all books from the bibliography

%\begin{refsection} % This is a custom heading for those references marked as "inproceedings" but not containing "keyword=france"
%\nocite{*}
%\printbibliography[sorting=chronological, type=inproceedings, title={international peer-reviewed conferences/proceedings}, notkeyword={france}, heading=bibheading]
%\end{refsection}

%\begin{refsection} % This is a custom heading for those references marked as "inproceedings" and containing "keyword=france"
%\nocite{*}
%\printbibliography[sorting=chronological, type=inproceedings, title={local peer-reviewed conferences/proceedings}, keyword={france}, heading=bibheading]
%\end{refsection}

%\printbibsection{misc}{other publications} % Print all miscellaneous entries from the bibliography

%\printbibsection{report}{research reports} % Print all research reports from the bibliography

%----------------------------------------------------------------------------------------

\end{document}