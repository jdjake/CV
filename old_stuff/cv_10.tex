%%%%%%%%%%%%%%%%%%%%%%%%%%%%%%%%%%%%%%%%%
% Friggeri Resume/CV
% XeLaTeX Template
% Version 1.2 (3/5/15)
%
% This template has been downloaded from:
% http://www.LaTeXTemplates.com
%
% Original author:
% Adrien Friggeri (adrien@friggeri.net)
% https://github.com/afriggeri/CV
%
% License:
% CC BY-NC-SA 3.0 (http://creativecommons.org/licenses/by-nc-sa/3.0/)
%
% Important notes:
% This template needs to be compiled with XeLaTeX and the bibliography, if used,
% needs to be compiled with biber rather than bibtex.
%
%%%%%%%%%%%%%%%%%%%%%%%%%%%%%%%%%%%%%%%%%

\documentclass{friggeri-cv} % Add 'print' as an option into the square bracket to remove colors from this template for printing

% \addbibresource{bibliography.bib} % Specify the bibliography file to include publications

\begin{document}

\header{jacob}{denson}{Computing Scientist and Mathematician} % Your name and current job title/field

%----------------------------------------------------------------------------------------
%	SIDEBAR SECTION
%----------------------------------------------------------------------------------------

\begin{aside} % In the aside, each new line forces a line break
\section{Contact Information}
%4212 109 Street
%Edmonton, AB, Canada
%T6J2S8
%~
%(780) 433-4544
%+1 (780) 862-7414
%~
Additional Contact Info Redacted For Web Version
\href{mailto:denson@ualberta.ca}{denson@ualberta.ca}
\section{Websites}
\href{https://github.com/jdjake}{{\bf Github Profile}: jdjake}
~
\href{http://stackoverflow.com/users/2601483/jacob-denson}{{\bf Stack Overflow Profile:} jacob-denson}
~
%\href{http://webdocs.cs.ualberta.ca/~denson/}{webdocs.cs.ualberta.ca/$\sim$ denson/ (Work in Progress)}
\section{Languages}
English, Basic German,
Python, Perl, C++, C, C\#, Matlab, HTML, Latex (This resume is proof!)
\section{Mathematics}
Linear/Abstract Algebra, Real \& Complex Analysis, Measure Theory, Functional Analysis, Topology, Smooth Manifolds, Mathematical Logic, Elementary Differential Equations, Stochastic Processes \& Brownian Motion, Category Theory
\section{Interests}
Computer Vision and Computational Geometry, Dynamical Systems and Ergodic Theory.
\end{aside}

%----------------------------------------------------------------------------------------
%   KEY SKILLS
%----------------------------------------------------------------------------------------

\section{Summary}

My adept knowledge of computing science and mathematics have been a solid aid to many groups. With my work on data-consolidation, Microsoft has cut partner pickup times by 80\%, saving money for the company and making the partner relationship more pleasant. My work on Cognate identification was crucial to the regeneration process of the near-extinct Totonac languages. My technical competency, enhanced by my strong experience in competitive programming, will add crucial knowledge and experience to your team.

%----------------------------------------------------------------------------------------
%	WORK EXPERIENCE SECTION
%----------------------------------------------------------------------------------------

\section{Experience}

\subsection{Summer Internships}

\begin{entrylist}

%------------------------------------------------

\entry
{2015}
{MICROSOFT}
{Redmond, Washington}
{\emph{Universal Store Data Cleansing} \\
Developed algorithms for data linkage. Utilizing various data-cleansing methods together with the Azure and Bing data-analysis packages, cleansed Microsoft's business partner database, removing redundant info, reducing database entries by 20\%. My manager for this project was Aman Kansal (\href{mailto:Kansal@microsoft.com}{Kansal@microsoft.com}). I also worked off-hours with a group of interns to send robot adventurers around the world (\href{http://www.projectatlas.ms/}{http://www.projectatlas.ms/}), and organized weekly talk sessions!}

%------------------------------------------------

\entry
{2014}
{UNIVERSITY OF ALBERTA}
{Edmonton, Alberta}
{\emph{Natural Language Processing and Cognate Identification} \\
Worked with the NLP group at the University of Alberta to develop cognate recognition algorithms. Successfully pushed to create a centralized database for storing cognate information, simplifying the learning process. This program was successfully used by linguists at the University of Alberta to understand the Totonac language group. Garrett Nicolai supervised the project (\href{mailto:Nicolai@ualberta.ca}{Nicolai@ualberta.ca}).}

%------------------------------------------------

\entry
{2013}
{UNIVERSITY OF ALBERTA}
{Edmonton, Alberta}
{\emph{Reinforcement Learning GAMES group} \\
Implemented efficient abstraction algorithms to create a Sokoban solver for the RLAI group at the University of Alberta, under mentor Harm Van Seijen (\href{mailto:Harm.Van.Seijen@gmail.com}{Harm.Van.Seijen@gmail.com}).}

%------------------------------------------------

\end{entrylist}

\pagebreak[2]
\subsection{Additional Work \& Experience}

\begin{entrylist}

\entry
{2015}
{UNIVERSITY OF ALBERTA}
{Edmonton, Alberta}
{\emph{`Tangible Introduction To Computing Science' Teaching Assistant} \\
Advised students in the honours stream of Computing Science who were taking CMPUT 275, a class which introduced students to basic algorithmics, such as asymptotic analysis, divide and conquer, dynamic programming, and such. Led office hours weekly and marked assignments.}

%------------------------------------------------

\entry
{2013-Now\ \ \ \ \ }
{Competitive Programming club}
{}
{\emph{Competitor} \\
Strong Competitor in Competitive Programming. Won the Microsoft 2014 Coding for Cash competition, placed 4th in the Alberta Collegiate programming contest in 2014 and 2015. Coached by Zachary Friggstadt (\href{mailto:zacharyf@ualberta.ca}{zacharyf@ualberta.ca}), ACM world finalist.}

\end{entrylist}

%----------------------------------------------------------------------------------------
%   EDUCATION SECTION
%----------------------------------------------------------------------------------------

\section{Education}

\begin{entrylist}

%------------------------------------------------

\entry
{2013-2017\ \ \ \ \ \ \ }
{Bachelors {\normalfont in Computing Science}}
{The University of Alberta}

%------------------------------------------------

\entry
{2011-2013}
{International Baccalaureate {\normalfont High School Diploma}}
{Harry Ainlay High School}

%------------------------------------------------

\end{entrylist}

%%----------------------------------------------------------------------------------------
%%	AWARDS SECTION
%%----------------------------------------------------------------------------------------
%
%\pagebreak[2]
%\section{Awards}
%
%\begin{entrylist}
%
%%------------------------------------------------
%
%\entry
%{2014}
%{Jason Lang Scholarship}
%{Alberta Scholarships}
%{Awarded to students Alberta post-secondary students continuing full-time in undergraduate programs with outstanding academic achievements.}
%
%------------------------------------------------
%
%\entry
%{2013}
%{Academic Excellence Scholarship}
%{University of Alberta}
%{Awarded to students with superior academic achievement entering the first year of an undergraduate degree program at the University of Alberta.}
%
%%------------------------------------------------
%
%\entry
%{2013}
%{Faculty of Science Academic Excellence Scholarship}
%{University of Alberta}
%{Awarded annually on the basis of superior academic achievement to students entering the first year of an undergraduate degree program in the Faculty of Science at the University of Alberta.}
%
%------------------------------------------------
%
%\entry
%{2013}
%{Alexander Rutherford Achievement Scholarship}
%{Alberta Scholarships}
%{To recognize and reward academic achievement at the senior high school level and to encourage students to pursue post-secondary studies.}
%
%%------------------------------------------------
%
%\end{entrylist}

%----------------------------------------------------------------------------------------
%	COMMUNICATION SKILLS SECTION
%----------------------------------------------------------------------------------------

\pagebreak[2]
\section{Talks} % \& Publications}

\begin{entrylist}

%------------------------------------------------

\entry
{2016}
{CUMC Undergraduate Conference}
{University of Victoria, Vancouver Island}
{`On Molecular Gases and the Natural Numbers', a quick, twenty-minute talk introduce the subject of Ergodic theory to undergraduate students, and emphasizing its relation to a variety of problems in mathematics, emphasizing number theory.}

%------------------------------------------------

\entry
{2015}
{Microsoft Intern Talks}
{Microsoft Campus, Redmond}
{Presented my talk on category theory, shortened to a 20 minutes talk, and edited to reduce mathematical prerequisites and to emphasize the practical uses for the average programmer. Organized talks over my internship to enable interns to share their knowledge with the group.}

%------------------------------------------------

\entry
{2015}
{Honours Computing Science Seminar}
{University of Alberta}
{`Category Theory and its relation to Computing Science', an hour-long talk introducing the subject to Honours computing scientists and emphasizing its relation to the Curry Howard isomorphism.}

%------------------------------------------------

\entry
{2014}
{NLP Research Group}
{University of Alberta}
{`Cognates for Reconstruction of Native American Language groups', a twenty minute talk emphasizing my work over the summer and explaining the organization method and SVM classification method for identifying cognates.}

%------------------------------------------------

\entry
{2013}
{RLAI Tea Time Talks}
{University of Alberta}
{`Room Abstraction in Sokoban', a 15 minute talk introducing the game of Sokoban, its combinatorial issues, and room abstraction as an aid to attacking the game.}

\end{entrylist}

%----------------------------------------------------------------------------------------
%	INTERESTS SECTION
%----------------------------------------------------------------------------------------

%\section{Interests}

%----------------------------------------------------------------------------------------
%	PUBLICATIONS SECTION
%----------------------------------------------------------------------------------------

%\section{publications}

%\printbibsection{article}{article in peer-reviewed journal} % Print all articles from the bibliography

%\printbibsection{book}{books} % Print all books from the bibliography

%\begin{refsection} % This is a custom heading for those references marked as "inproceedings" but not containing "keyword=france"
%\nocite{*}
%\printbibliography[sorting=chronological, type=inproceedings, title={international peer-reviewed conferences/proceedings}, notkeyword={france}, heading=bibheading]
%\end{refsection}

%\begin{refsection} % This is a custom heading for those references marked as "inproceedings" and containing "keyword=france"
%\nocite{*}
%\printbibliography[sorting=chronological, type=inproceedings, title={local peer-reviewed conferences/proceedings}, keyword={france}, heading=bibheading]
%\end{refsection}

%\printbibsection{misc}{other publications} % Print all miscellaneous entries from the bibliography

%\printbibsection{report}{research reports} % Print all research reports from the bibliography

%----------------------------------------------------------------------------------------

\end{document}